% !TEX TS-program = pdflatex

\documentclass{article}
\usepackage{amsmath,amssymb,amsthm}
\usepackage{caption,float}

\title{Projects for Numerical methods for Machine Learning}
\author{Mehdi Jalili}

\begin{document}
\maketitle

\section{Face recognition using SVD}
You can consider images as vectors $ \mathbb{R}^{n_i} $, hence a database of images of $ n_p $ persons in $ n_e $ different expressions can be represented by $ n_p $ different matrices $ A_p \in \mathbb{R}^{n_i \times n_e} (p=1,\ldots, n_p)$.


The idea is to “model” the variation of faces of each person in the
training set using an orthogonal basis of the subspace of $ \mathbb{R}^{n_i} $ spanned by the columns of
$ A_p $ . This basis can be computed using the SVD, which enables us to write $ A_p $ as a sum
of rank-one matrices:

\begin{align}
	A_p = \sum\limits_{i=1}^{n_e} \sigma_i^{(p)} u_i^{(p)}(v_i^{(p)})^T,\quad p = 1,\cdots,n_p.
\end{align}

\end{document}
